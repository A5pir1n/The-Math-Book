\documentclass[main.tex]{subfiles}
\begin{document}
\section{Systems of Linear Equations}
Suppose $\mathbb{F}$ is a field. We consider the problem of finding $n$ scalars (elements of $F$ ) $x_1, \ldots, x_n$ which satisfy the conditions

\begin{equation}
\label{eq:system_of_m_linear_equations_in_n_unkowns}
\begin{alignedat}{3}
A_{11}x_1               &\, + \,& A_{12}x_2               & +\cdots+ & A_{1 n} x_n             &=\,&y_1 \\
A_{21}x_1               &\, + \,& A_{22}x_2               & +\cdots+ & A_{2 n} x_n             &=\,&y_2 \\
\vdotswithin{A_{11}x_1} &       & \vdotswithin{A_{12}x_1} &          & \vdotswithin{A_{1n}x_n} &   &\vdotswithin{y_1} \\
A_{m1}x_1               &\, + \,& A_{m2}x_2               & +\cdots+ & A_{m n} x_n             &=\,&y_m
\end{alignedat}
\end{equation}

where $y_1, \ldots, y_m$ and $A_{i j}, 1 \leq i \leq m, 1 \leq j \leq n$, are given elements of $F$. We call \ref{eq:system_of_m_linear_equations_in_n_unkowns} a system of $m$ linear equations in $n$ unknowns. Any $n$-tuple $\left(x_1, \ldots, x_n\right)$ of elements of $F$ which satisfies each of the equations in \ref{eq:system_of_m_linear_equations_in_n_unkowns} is called a \textbf{solution} of the system. If $y_1=y_2=\cdots=$ $y_m=0$, we say that the system is \hypertarget{homogeneous}{\textbf{homogeneous}}\index{homogeneous}, or that each of the equations is homogeneous. 

\subsection*{Technique of Elimination in Solving System of Linear Equations: A need of formal process}
In elementary algebra, we have been introduced the process of ``eliminating unknowns'' to solve system of linear equations. That is, by multiplying equations by scalars and then adding to produce equations in which some of the $x_i$ were not present. To illustrate this technique, we consider a homogeneous system: 
$$\begin{aligned} 2 x_1-x_2+x_3 & =0 \\ x_1+3 x_2+4 x_3 & =0\end{aligned}$$
%%TODO: properly align these
If we add $(-2)$ times the second equation to the first equation, we obtain
$$
-7 x_2-7 x_3=0
$$
or, $x_2=-x_3$. If we add 3 times the first equation to the second equation, we obtain
$$
7 x_1+7 x_3=0
$$
or, $x_1=-x_3$. So we conclude that if $\left(x_1, x_2, x_3\right)$ is a solution then $x_1=x_2=$ $-x_3$. We can let it be any number $a = x_1=x_2=$ $-x_3$ where $a \in \mathbb{F}$. Thus the set of solutions consists of all triples $(-a,-a, a)$.

Now, if we add a third equation to our system of equations and if this third equation is a multiple of any of the two, we see that it does not add any new information to our solution. For example, if we have 
$$\begin{aligned} 2 x_1-x_2+x_3 & =0 \\ 
x_1+3 x_2+4 x_3 & =0\\
4 x_1-2x_2+2x_3 & =0.
\end{aligned}$$
When we try to add multiples of the third equation to the first one to eliminate any unknown: if we add $(-2)$ times the first equation to the third, we just get $0 = 0$. If we add $(-4)$ times the second equation to the third equation, we get
$$
-14x_1+16 x_3=0,
$$
or, $x_2=-x_3$, the same as if we added $(-2)$ times the second equation to the first equation. Other processes will result similarly. 

However, if the third equation is not a multiple of the first or the second equation. For example, let it be 
$$\begin{aligned} 2 x_1-x_2+x_3 & =0 \\ 
x_1+3 x_2+4 x_3 & =0\\
x_1+2x_2+2x_3 & =0.
\end{aligned}$$ 
Since we already have solution in the form of $(-a, -a, a)$ from the first two equations, we plug it in to the third one we get: 
$$-a -2a + 2a = 0,$$
or, $a = 0$, meaning $x_1 = x_2 = x_3 = 0$. 

Here, we intend to formalize this process so that we understand why it works, and how we can solve a system in a systematic manner. 



\end{document}