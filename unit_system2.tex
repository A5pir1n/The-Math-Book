\documentclass[main.tex]{subfiles}
\begin{document}

This article comes form Professor ALAN L. MYERS at \url{https://www.seas.upenn.edu/~amyers/NaturalUnits.pdf}


% Insert bastract
\begin{Abstract}
\begin{changemargin}{1cm}{1cm}
The international system (SI) of units based on the kilogram, meter, and second is inconvenient for computations involving the masses of stars or the size of a galaxy. Papers about quantum gravity express mass, length, and time in terms of energy, usually powers of $\mathrm{GeV}$. Geometrized units, in which all units are expressed in terms of powers of length are also prevalent in the literature of general relativity. Here equations are provided for conversions from geometrized or natural units to SI units in order to make numerical calculations.
\end{changemargin}
\end{Abstract}

\section{International System of Units}
The International System of Units (SI) is based on the meter-kilogram-second (MKS) system of units. A first course in any field of science or engineering usually begins with a discussion of SI units. Given an equation, students learn to check instinctively the units of the various terms for consistency. For example, in SI units for Newton's law of universal gravitation
\begin{equation}\label{eq:Newton's_law_of_gravitation}
F=\frac{G m M}{r^2}
\end{equation}
$m$ and $M$ are the masses of the two bodies in kilograms, $r$ is the distance between their centers in meters, $F$ is the force of attraction in newtons $\left(1 \mathrm{~N}=1 \mathrm{~kg} \mathrm{~m} \mathrm{~s}^{-2}\right)$, and $G=$ Newton's constant $=6.6743 \times 10^{-11} \mathrm{~m}^3 \mathrm{~kg}^{-1} \mathrm{~s}^{-2}$. This equation satisfies the consistency test because the terms on both sides of the equation have the same units $\left(\mathrm{kg} \mathrm{m} \mathrm{s}^{-2}\right)$.

Cosmologists toss the SI system of units out the window so that every variable is expressed in powers of energy. The equations are simplified by the absence of constants including Newton's constant $(G)$ and the speed of light $(c)$. However, for a person educated in the SI system of units, the lingo of "natural units" used by cosmologists is confusing and seems (incorrectly) to display a casual disregard of the importance of units in calculations.
First let us examine units term-by-term in the Einstein equation of general relativity
$(1)$
$$
R_{\mu \nu}-\frac{1}{2} R g_{\mu \nu}+\Lambda g_{\mu \nu}=\frac{8 \pi G}{c^4} T_{\mu \nu}
$$

written in SI units. Popular articles and most textbooks on general relativity introduce this equation without discussing its units. The energy-momentum tensor $T_{\mu \nu}$ has units of energy density $\left(\mathrm{J} \mathrm{m}^{-3}\right)$ or, equivalently, momentum flux density $\left(\mathrm{kg} \mathrm{m}^{-1} \mathrm{~s}^{-2}\right)$. Multiplication of the units of $T_{\mu \nu}\left(\mathrm{kg} \mathrm{m}^{-1} \mathrm{~s}^{-2}\right)$ by the units of $G / c^4\left(\mathrm{~s}^2 \mathrm{~kg}^{-1} \mathrm{~m}^{-1}\right)$ yields units of $\mathrm{m}^{-2}$ for the RHS of the equation. The terms on the LHS must have the same units $\left(\mathrm{m}^{-2}\right)$. The metric tensor $\left(g_{\mu \nu}\right)$ is dimensionless. The Riemann tensor $(R)$ is the second derivative with respect to distance of the metric tensor and therefore has units of $\mathrm{m}^{-2}$. The index-lowered forms of the Riemann tensor, $R_{\mu \nu}$ and $R$, have units of $\mathrm{m}^{-2}$ so the units are the same on both sides of the equation, as required.

\section{Natural Units}
So-called natural units are used almost exclusively in cosmology and general relativity. In order to read the literature, it is necessary to learn how to write equations and perform calculations in natural units. In the version of natural units used in cosmology, four fundamental constants are set to unity:
$$
c=\hbar=\epsilon_{\circ}=k_B=1
$$

where
$$
\begin{aligned}
c & =\text { speed of light }=2.9979 \times 10^8 \mathrm{~m} / \mathrm{s} \\
\hbar & =\text { reduced Planck constant }=1.0546 \times 10^{-34} \mathrm{~J} \mathrm{~s}^{-1} \\
\epsilon_{\circ} & =\text { electric constant }=8.8542 \times 10^{-12} \mathrm{~A}^2 \mathrm{~s}^4 \mathrm{~kg}^{-1} \mathrm{~m}^{-3} \\
k_B & =\text { Boltzmann constant }=1.3806 \times 10^{-23} \mathrm{~J} \mathrm{~K}^{-1}
\end{aligned}
$$
As a consequence of these definitions, $1 \mathrm{~s}=2.9979 \times 10^8 \mathrm{~m}$ and $1 \mathrm{~s}^{-1}=1.0546 \times 10^{-34} \mathrm{~J}$. Length and time acquire the units of reciprocal energy; energy and mass have the same units. Any kinematical variable with SI units of $\left(\mathrm{kg}^\alpha \mathrm{m}^\beta \mathrm{s}^\gamma\right)$ may be expressed in SI units:
\begin{equation}
(E)^{\alpha-\beta-\gamma} \hbar^{\beta+\gamma} c^{\beta-2 \alpha}
\end{equation}
where $E$ in an arbitrarily chosen energy unit.


To see the unit conversion more easily, it is often easier to remember the fundamental physics for mule tint involve these fundamental constant For example, we know $E=m c^2$ so $[E]=[m][c]^2=\operatorname{kg}[c]^2$ and then $\mathrm{kg}=\frac{[E]}{[c]^2} \Rightarrow \mathrm{kg}^\alpha=\frac{[E]^\gamma}{[C]^{2 n}}$. Next, we use $E=\hbar \omega$ to get $[E]=[\hbar][\omega]=[\hbar] s^{-1}$ (treating radian unitless) $\Rightarrow S=\frac{[\hbar]}{[E]}$ or $S^\gamma=\frac{[\hbar]^r}{[E]^\gamma}$. The last one is easy $v=\frac{d s}{d t} \Rightarrow[V]=\frac{[s]}{[t]}$ $\Rightarrow[c]=\frac{m}{s}$. so $m=s \cdot[c]=\frac{[\hbar]}{[E]}[c]$. $\Rightarrow m^\beta=\frac{[\hbar]^\beta}{[E]^\beta}[c]^\beta$. Combining all the units: $\mathrm{kg}^\alpha \mathrm{m}^\beta \mathrm{s}^\gamma = \frac{[E]^\alpha}{[c]^{2 \alpha}} \frac{[\hbar]^\beta}{[E]^\beta}[c]^\beta \cdot \frac{[\hbar]^\gamma}{[E]^\gamma} = [E]^{\alpha-\beta-\gamma} \hbar^{\beta+\gamma} c^{\beta-2 \alpha}$



A popular choice is $\mathrm{GeV}\left(1 \mathrm{GeV}=1.6022 \times 10^{-10} \mathrm{~J}\right)$. Setting the constants $\hbar$ and $c$ equal to unity gives natural units of $\mathrm{GeV}^{\alpha-\beta-\gamma}$. Given natural units of $\mathrm{GeV}^{\alpha-\beta-\gamma}$, the desired SI unit can always be recovered by multiplying by the conversion factor $\left(\hbar^{\beta+\gamma} c^{\beta-2 \alpha}\right)$

Consider, for example, momentum with SI units of $\left(\mathrm{kg} \mathrm{m} \mathrm{s}^{-1}\right)$. For $\alpha=1, \beta=1$, and $\gamma=-1$, the natural unit is $\mathrm{GeV}^{\alpha-\beta-\gamma}=\mathrm{Gev}^1$ and the conversion factor is $\mathrm{c}^{-1}$. Conversion from natural to SI units gives:
$$
1 \mathrm{GeV}=\frac{E}{c}=\frac{1.6022 \times 10^{-10} \mathrm{~J}}{2.9979 \times 10^8 \mathrm{~m} \mathrm{~s}^{-1}}=5.3444 \times 10^{-19} \mathrm{~kg} \mathrm{~m} \mathrm{~s}
$$
Table 1 provides conversion factors for some of the variables encountered in cosmology. Energy, mass, and momentum have the same natural units. Velocity, angular momentum, and charge are dimensionless. Note that pressure (force per unit area) has the same units as energy density in both systems of units, as it must.

In the system of natural units, the factors in Table 1 are unity by definition. To make conversions, cosmologists multiply or divide by the factors $\hbar$ and $c$ with appropriate exponents to obtain the units desired. This seems like a trial-and-error procedure! Rules for converting units for SI to natural units, and for the reverse conversion, are given in Table 1 .

\begin{tabular}{|c|c|c|c|c|c|}
\hline \multirow{2}{*}{$\begin{array}{l}\text { Variable } \\
\text { mass }\end{array}$} & \multirow{2}{*}{$\begin{array}{l}\text { SI Unit } \\
\mathrm{kg}\end{array}$} & \multirow{2}{*}{$\begin{array}{l}\text { Natural Unit } \\
\mathrm{E}\end{array}$} & \multirow{2}{*}{$\frac{\text { Factor }}{c^{-2}}$} & \multicolumn{2}{|c|}{ Natural unit $\rightarrow$ SI unit } \\
\hline & & & & $1 \mathrm{GeV}$ & $\rightarrow 1.7827 \times 10^{-27} \mathrm{~kg}$ \\
\hline length & $\mathrm{m}$ & $\mathrm{E}^{-1}$ & $\hbar c$ & $1 \mathrm{GeV}^{-1}$ & $\rightarrow 1.9733 \times 10^{-16} \mathrm{~m}$ \\
\hline time & $\mathrm{s}$ & $\mathrm{E}^{-1}$ & $\hbar$ & $1 \mathrm{GeV}^{-1}$ & $\rightarrow 6.5823 \times 10^{-25} \mathrm{~s}$ \\
\hline energy & $\mathrm{kg} \mathrm{m}^2 \mathrm{~s}^{-2}$ & $\mathrm{E}$ & 1 & $1 \mathrm{GeV}$ & $\rightarrow 1.6022 \times 10^{-10} \mathrm{~J}$ \\
\hline momentum & $\mathrm{kg} \mathrm{m} \mathrm{s} \mathrm{m}^{-1}$ & $\mathrm{E}$ & $c^{-1}$ & $1 \mathrm{GeV}$ & $\rightarrow 5.3444 \times 10^{-19} \mathrm{~kg} \mathrm{~m} \mathrm{~s} \mathrm{~s}^{-1}$ \\
\hline velocity & $\mathrm{m} \mathrm{s}^{-1}$ & dimensionless & $c$ & 1 & $\rightarrow 2.9979 \times 10^8 \mathrm{~m} \mathrm{~s}^{-1}$ \\
\hline angular momentum & $\mathrm{kg} \mathrm{m}^2 \mathrm{~s}^{-1}$ & dimensionless & $\hbar$ & 1 & $\rightarrow 1.0546 \times 10^{-34} \mathrm{~J} \mathrm{~s}$ \\
\hline area & $\mathrm{m}^2$ & $\mathrm{E}^{-2}$ & $(\hbar c)^2$ & $1 \mathrm{GeV}^{-2}$ & $\rightarrow 3.8938 \times 10^{-32} \mathrm{~m}^2$ \\
\hline force & & $\mathrm{E}^2$ & $(\hbar c)^{-1}$ & $1 \mathrm{GeV}^2$ & $\rightarrow 8.1194 \times 10^5 \mathrm{~N}$ \\
\hline energy density & $\mathrm{kg} \mathrm{m}^{-1} \mathrm{~s}^{-2}$ & $\mathrm{E}^4$ & $(\hbar c)^{-3}$ & $1 \mathrm{GeV}^4$ & $\rightarrow 2.0852 \times 10^{37} \mathrm{~J} \mathrm{~m}^{-3}$ \\
\hline charce & $\mathrm{C}=\mathrm{A} \cdot \mathrm{s}$ & dimensionless & 1 & 1 & $\rightarrow 52909 \times 10^{-19} \mathrm{C}$ \\
\hline
\end{tabular}

The entry in Table 1 for charge requires an explanation. The dimensionless fine-structure constant $(\alpha)$ is:
$$
\alpha=\frac{e^2}{4 \pi \epsilon_{\circ} \hbar c}=0.0072974
$$

The elementary charge $e=1.6022 \times 10^{-19} \mathrm{C}$ and the value of $\alpha$ is calculated from the set of constants $\left\{e, \epsilon_{\circ}, \hbar, c\right\}$. In natural units, $\epsilon_{\circ}=\hbar=c=1$ and the dimensionless elementary charge $e$ is:
$$
e=\sqrt{4 \pi \alpha}=0.30282
$$
so that
$$
0.30282=1.6022 \times 10^{-19} \mathrm{C} \quad \Longrightarrow \quad 1=5.2909 \times 10^{-19} \mathrm{C}
$$

Remark: Notice that the fine structure is the constant here whereas the elementary charge is defined using the fine structure constant $\alpha$. 

In natural units, Eq. (1) becomes:
$$
R_{\mu \nu}-\frac{1}{2} R g_{\mu \nu}+\Lambda g_{\mu \nu}=8 \pi G T_{\mu \nu}
$$
The Planck mass is
$$
m_p=\sqrt{\frac{\hbar c}{G}}=2.1764 \times 10^{-8} \mathrm{~kg}
$$
In natural units $m_p=1.2209 \times 10^{19} \mathrm{GeV}$. Replacement of Newton's gravitational constant $G$ in Einstein's equation with the Planck mass gives
$$
R_{\mu \nu}-\frac{1}{2} R g_{\mu \nu}+\Lambda g_{\mu \nu}=\frac{8 \pi}{m_p^2} T_{\mu \nu}
$$
Continuing with natural units, the energy-momentum tensor has units of energy density or $\mathrm{GeV}^4$ and the Planck mass has units of $\mathrm{GeV}$. The RHS of the equation therefore has units of $\mathrm{GeV}^2$. $\mathrm{On}^2$ the LHS of the equation, the metric tensor $g_{\mu \nu}$ is dimensionless so the Ricci tensor $R_{\mu \nu}$, Ricci scalar $R$, and the cosmological constant $\Lambda$ all have natural units of $\mathrm{GeV}^2$, or mass squared since energy and mass are equivalent.

As an exercise in the manipulation of natural units, consider the cosmological constant $(\Lambda)$ which is frequently characterized with the same units as the energy-momentum tensor $G_{\mu \nu}$ and called the energy density of a vacuum $\left(\rho_{v a c}\right)$ :
$$
\rho_{v a c} \approx 3 \times 10^{-47} \mathrm{GeV}^4
$$
Transfer of the quantity from the RHS to the LHS of Eq. (3) requires multiplication by the factor of $8 \pi / m_p^2$ :
$$
\Lambda=\rho_{v a c}\left(\frac{8 \pi}{m_p^2}\right)=\frac{(8 \pi)\left(3 \times 10^{-47} \mathrm{GeV}^4\right)}{\left(1.2209 \times 10^{19} \mathrm{GeV}\right)^2}=5.06 \times 10^{-84} \mathrm{GeV}^2
$$
These natural units for $\Lambda$ may be converted to the SI unit of $\mathrm{m}^{-2}$ using the conversion factor in Table 1
$$
\Lambda=5.06 \times 10^{-84} \mathrm{GeV}^2\left(\frac{1 \mathrm{GeV}^{-2}}{3.8938 \times 10^{-32} \mathrm{~m}^2}\right)=1.3 \times 10^{-52} \mathrm{~m}^{-2}
$$
Conversion of the vacuum energy density from natural to SI units gives:
$$
\rho_{v a c}=3 \times 10^{-47} \mathrm{GeV}^4\left(\frac{2.0852 \times 10^{37} \mathrm{~J} \mathrm{~m}^{-3}}{1 \mathrm{GeV}^4}\right)=6.3 \times 10^{-10} \mathrm{~J} \mathrm{~m}^{-3}
$$
In terms of the the $m c^2$ energy of mass, $\rho_{v a c} \approx 4$ hydrogen atoms per cubic meter.

\end{document}