\documentclass[main.tex]{subfiles}
\begin{document}
\chapter{Analysis}
In mathematics, analysis usually refers to any of a broad family of fields that deals with a general theory of limits in the sense of convergence of sequences (or more generally of nets), particularly those fields that pursue developments that originated in “the calculus”, i.e., the theory of differentiation (differential calculus) and integration (integral calculus) of real and complex-valued functions. The classical foundation of this general subject is usually based on the idea that the real number system is describable as a complete ordered field, or more generally on the concept of metric spaces. Their distance functions allow to formalize concepts like continuity and convergence in terms of existence of sufficiently small open balls. Many concepts of this “epsilontic analysis” have equivalent formulations in terms of simple combinatorics of open subsets with respect to the metric topology of metric spaces, and this way the field of analysis has a large overlap with the field of topology, this is particularly true for functional analysis and the theory of topological vector spaces.

Analysis can also refer to other responses to the problem of founding these developments, especially “infinitesimal analysis” which admits infinitesimal quantities not found in the classical real number system and which takes various forms, for example the nonstandard analysis first introduced by Abraham Robinson, or “synthetic differential analysis” whose rigorous foundations were largely introduced by William Lawvere and other category theorists who, following the example of Alexander Grothendieck, consider nilpotent infinitesimals (instead of invertible ones à la Robinson) as a basis for understanding differentiation.


%Reference:: https://ncatlab.org/nlab/show/analysis
\end{document}