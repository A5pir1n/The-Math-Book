\documentclass{book}
\usepackage[utf8]{inputenc}
\usepackage{amsmath, amsthm}
\usepackage{physics} %This gives the option to use \norm{}
\usepackage{multirow} %allows merging rows

%This gives and option to combine \textbf{\textsc{...}}
\usepackage[T1]{fontenc}


% This command get rid of the first figure of the section number coming from the chapter. For example, changes chapter2 section 1 from 2.1 to 1. 
\renewcommand{\thesection}{\arabic{section}}

% \renewcommand{\thesubsection}{\arabic{subsection}}



\newtheorem{thm}{Theorem}[section]
\newtheorem{prop}{Proposition}[section]
\newtheorem{corollary}{Corollary}[theorem]
\newtheorem{lemma}{Lemma}[theorem]



\theoremstyle{remark}
\newtheorem*{note}{Note}
\newtheorem{remark}{Remark}[section]
\newtheorem{example}{Example}[section]

\theoremstyle{definition}
\newtheorem{definition}{Definition}[section]

\newcommand{\sgn}{\text{sgn\,}}
\newcommand{\inv}[1]{{#1}^{-1}}



\title{Combinitorics}
\author{Hongyu}
\date{\today}


\begin{document}

\section{Permutation}
\begin{definition}
A sequence $(k_1, k_2, \cdots k_n)$ of positive integers not exceeding $n$, with the property that no two of the $k_i$ are equal, is called a \textbf{permutation of degree n}. 

Since a finite sequence, or \textbf{\textit{n}-tuple}, is a function defined on the first $n$ positive integers, a permutation of degree $n$ may be defined as a one-to-one function from the set $\{1, 2, \cdots n \}$ onto it self. Such a function $\sigma$ corresponds to the $n$-tuple $(\sigma 1, \sigma 2, \cdots, \sigma n )$ ad is thus simply a rule for ordering $1, 2, \cdots, n$ in some well-defined way. 
\end{definition}

All possible number of permutation of degree $n$ is $n!$.

\section{Elementary combinatorics}

\begin{definition}
If we only consider rearrangements of $k$ elements taken a given set of size $n$, we call them \textbf{$k$-permutations of $n$}, which are different ordered arrangements of a $k$-element subset of an $n$-set. They are also known as \textbf{partial permutations} or as \textbf{sequences without repetition}.
\end{definition}

\begin{definition}
Another closely related concept is combination, the element of which is not ordered. 
\end{definition}

The formulae for the number of $k$-permutation of $n$ and $k$ combination of $n$ can be summarized as: 
\begin{table}[h!]
    \centering
    \begin{tabular}{|c|c|c|c|}
        \hline
         {} & {} & \multicolumn{2}{|c|}{Consider repetition} \\\cline{3-4}
         {} & {} & No & Yes \\\hline
         \multirow{2}{*}{Consider ordering} & No & $\frac{n!}{(n - r)!r!}$ & $\frac{(n + r - 1)!}{(n-1)!r!}$\\\cline{2-4}
         & Yes & $\frac{n!}{(n - r)!}$ & $n^r$\\\hline
         
    \end{tabular}
    \caption{Caption}
    \label{tab:my_label}
\end{table}
