\documentclass{article}
\usepackage[utf8]{inputenc}
\usepackage{amsmath, amsthm}


\newtheorem{theorem}{Theorem}[section]
\newtheorem{corollary}[theorem]{Corollary}
\newtheorem{lemma}[theorem]{Lemma}


\theoremstyle{remark}
\newtheorem{remark}[theorem]{Remark}
\newtheorem{example}[theorem]{Example}

\theoremstyle{definition}
\newtheorem{definition}{Definition}[section]



\title{Ordinary Differential Equations Notes}
\author{Hongyu}
\date{\today}

\begin{document}

\section{General Remarks}    
    
\begin{remark}
    Moving to a bigger space in which we keep track of everything that is relevant for the evolution of our system is called ‘working in \textbf{phase space}’: The phase space of a general $n$th order ODE is the space of all possible times, positions, and first $n - 1$ derivatives of the position. For a ball in flight we could take the phase space to simply be the position and velocity of the ball; more accurate treatment might include additional variables to keep track of how the ball is spinning etc. (TBM)
\end{remark}
    
\end{document}
    
