\documentclass[main.tex]{subfiles}
% \graphicspath{{\subfix{../figures/}}}
\begin{document}
\begin{example}
Let $F$ be a field and in $F^n$ let $S$ be the subset consisting of the vectors $\epsilon_1, \epsilon_2, \ldots, \epsilon_n$ defined by
$$
\begin{aligned}
\epsilon_1 & =(1,0,0, \ldots, 0) \\
\epsilon_2 & =(0,1,0, \ldots, 0) \\
& \hspace{1cm}\vdots \\
\epsilon_n & =(0,0,0, \ldots, 1) .
\end{aligned}
$$
Let $x_1, x_2, \ldots, x_n$ be scalars in $F$ and put $\alpha=x_1 \epsilon_1+x_2 \epsilon_2+\cdots+x_n \epsilon_n$. Then
$$
\quad \alpha=\left(x_1, x_2, \ldots, x_n\right) \text {. }
$$
This shows that $\epsilon_1, \ldots, \epsilon_n$ span $F^n$. Since $\alpha=0$ if and only if $x_1=$ $x_2=\cdots=x_n=0$, the vectors $\epsilon_1, \ldots, \epsilon_n$ are linearly independent. The set $S=\left\{\epsilon_1, \ldots, \epsilon_n\right\}$ is accordingly a basis for $F^n$. We shall call this particular basis the \textbf{standard basis} of $F^n$.
\end{example}
\end{document}