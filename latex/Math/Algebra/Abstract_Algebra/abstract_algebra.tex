\documentclass[main.tex]{subfiles}

\begin{document}
\chapter{Basic Algebra}
\section{Law of Composition/Binary Function}
A \textbf{law of composition} on a set $S$ is a rule for combining pairs $a, b$ of elements of $S$ to get another element, say $p$ in $s$. Examples of this include addition and multiplication on the set of real numbers. Formally, a law of composition is a binary function, a function of two variables. It is a map
$$ S \cross S \rightarrow S. $$
$S \cross S$ denotes the \textbf{product set}, whose elements are pairs $a,b$ of element of $S$. 

\begin{prop}
    Let an associative binary function be given on a set $S$. There is a unique way to define, for every integer $n$, a product of $n$ elements $a_1, \ldots, a_n$ of $S$, denoted temporarily by $\left[a_1 \cdots a_n\right]$, with the following properties:
    \begin{enumerate}[(i)]
        \item The product $\left[a_1\right]$ of one element is the element itself.
        \item The product $\left[a_1 a_2\right]$ of two elements is given by the binary function.
        \item For any integer $i$ in the range $1 \leq i<n, \quad\left[a_1 \cdots a_n\right]=\left[a_1 \cdots a_i\right]\left[a_{i+1} \ldots a_n\right]$.
    \end{enumerate}
\end{prop} 

\begin{proof}
We use induction on $n$. The product is defined by (i) and (ii) for $n \leq 2$, and it does satisfy (iii) when $n=2$. Suppose that the proposition is true for all of $2 < r \leq n - 1$, i.e. we have defined the product of $r$ elements when $2 < r \leq n-1$, and that it is the unique product satisfying (iii). We then uniquely define the product of $n$ elements by the rule
$$
\left[a_1 \cdots a_n\right]=\left[a_1 \cdots a_{n-1}\right]\left[a_n\right]
$$
where both terms on the right side are defined by our induction hypothesis. This can be a product if it exists, since it is (iii) when $i=n-1$. Now we must check (iii) for $i<n$ :
$$
\begin{array}{rlrl}
{\left[a_1 \cdots a_n\right]} & =\left[a_1 \cdots a_{n-1}\right]\left[a_n\right] & & \text { (our definition) } \\
& =\left(\left[a_1 \cdots a_i\right]\left[a_{i+1} \cdots a_{n-1}\right]\right)\left[a_n\right] & & \text { (induction hypothesis) } \\
& =\left[a_1 \cdots a_i\right]\left(\left[a_{i+1} \cdots a_{n-1}\right]\left[a_n\right]\right) & & \text { (associative law) } \\
& =\left[a_1 \cdots a_i\right]\left[a_{i+1} \cdots a_n\right] & & \text { (induction hypothesis). }
\end{array}
$$
This completes the proof. We can drop the brackets from now on and denote the product by $a_1 \cdots a_n$
\end{proof}

\begin{definition}
    An \textbf{identity} of a binary function is an element $e$ of $S$ such that
    $$ ea = ae = a, \text{for all } a \text{ in } S $$.
\end{definition}
The identity is unique. 
%Expandable-Trivial_proof
\begin{proof}
Suppose $e$ and $e^{\prime}$ are two such elements, then since $e$ is an identity, $e e^{\prime}=e^{\prime}$, and since $e^{\prime}$ is an identity, $e=e e^{\prime}$. Thus $e=e e^{\prime}=e^{\prime}$.
\end{proof}
The identity is denoted differently for different operations. For matrix multiplication, the identity is denoted as $I$ such that $$ IA = AI = A, \text{for all } A \text{ in a set of $n\times n$ matrices } M. $$ For a set of maps $T \rightarrow T$, it is the identity map $$i(a) = a \text{ for all $a$ in a set $T$. }$$ For multiplication of real number is it denoted as 1: $$1a = a1 = a, \text{ for all $a$ in $\mathbb{R}$ }.$$

\chapter{Groups}
\section{Groups}
\begin{definition}
A \textbf{group} is a set $G$ together with a law of composition that has the following properties:
\begin{enumerate}
    \item The law of composition is associative: $(a b) c=a(b c)$ for all $a, b, c$ in $G$.
    \item $G$ contains a identity element $e$ , such that $ea=a$ and $ae=a$ for all $a$ in $G$.
    \item Every element $a$ of $G$ has an inverse, an element $b$ such that $a b=1$ and $b a=1$.
\end{enumerate}

An \textbf{abelian group} is a group whose law of composition is commutative.

We use the notation $ab$ to denote the product of $a$ and $b$ with some law of composition, which can be addition of real numbers, multiplication of real numbers, matrices, etc. 


\end{definition}
\begin{example}
Some common examples of groups include: 
\begin{enumerate}
    \item $\mathbb{Z}^{+}: \quad$ the set of integers, with addition as its binary operation is an abelian group - the additive group of integers
    \item The set of nonzero real numbers forms an abelian group under multiplication.
    \item The set of all real numbers forms an abelian group under addition.
    \item The set of invertible $n \times n$ matrices, the general linear group, is a very important group in which the law of composition is matrix multiplication. It is not abelian unless $n=1$.
\end{enumerate}  
\end{example}

\subsection{Exercise}
\begin{enumerate}
    \item Let $x, y, z$, and $w$ be elements of a group $G$.
    (a) Solve for $y$, given that $x y z^{-1} w=1$.
    (b) Suppose that $x y z=1$. Does it follow that $y z x=1$ ? Does it follow that $y x z=1$ ?
    \begin{solution}
    \begin{enumerate}[(a)]
        \item $y = x^{-1} x y z^{-1} z = x^{-1} x y z^{-1} w w^{-1} z=x^{-1}1w^{-1} z = x^{-1}w^{-1} z$
        \item Yes, $y z x=(x^{-1}x) y z x= x^{-1}(x y z) x=x^{-1}(1) x=1$. No, it isn't necessarily the case that $y x z=1$. For example, in $S_3$ letting $x$ be cyclic permutation $(\mathbf{123})$ and $y$ be the transposition $(\mathbf{12})$, we have $x \cdot y \cdot x y=1$, but $y \cdot x \cdot x y=x \neq 1$.
    \end{enumerate}
    \end{solution}
\end{enumerate}

\section{Subgroups of the Additive Group of Integers $(\mathbb{Z}, +)$}
We introduce some elementary number theory in the language of additive group $(\mathbb{Z}, +)$ of integers. To begin, we list the definitions for a subgroup when additive notation is used in the group. 
\begin{definition}
A subset $S$ of a group $G$ with law of composition written additively is a subgroup if it has these properties:
\begin{enumerate}
    \item Closure: If $a$ and $b$ are in $S$, then $a+b$ is in $S$.
    \item Identity: 0 is in $S$.
    \item Inverses: If $a$ is in $S$ then $-a$ is in $S$.
\end{enumerate}
\end{definition}
\chapter{Rings}
\section{One}
\begin{definition}
A \textbf{ring} is a set $K$, together with two operations $(x, y) \rightarrow x + y)$ and $(x, y) \rightarrow xy$ satisfying: 
\begin{enumerate}[label=\Alph*]
    \item $K$ is a commutative group under the operation $(x, y) \rightarrow x + y)$ (K is a commutative group under addition);
    \item $(xy)z = x(yz)$ (multiplication is associative);
    \item $x(y + z)  = xy + xz$; $(y + z)x = yx + zx$ (distributive laws hold.
\end{enumerate}
If $xy = yx$ for all $x$ and $y$ in $K$, we say that the ring $K$ is \textbf{commutative}. If there is an element $1$ in $K$ such that $1x = x1 = x$ for each $x$, $K$ is said to be a $ring with identity$, and $1$ is called the \textbf{identity} for $K$.  
\end{definition}

\chapter{Fields}
\section{Field}
\begin{definition}
    

If we have a set $F$ of objects $x, y, z \cdots$ and two operations on the elements of $F$ as follows. The first operation, called addition, associate with each pair of elements $x, y$ in $F$ an element $(x + y)$ in $F$; the second operation called multiplication, associate with each pair of elements $x, y$ in $F$ an element $xy$ in $F$. The set $F$, together with these two operations is called a field if they satisfy the conditions (1) - (9) below. 
\begin{enumerate}
    \item Addition is commutative,
$$
x+y=y+x,
$$
for all $x$ and $y$ in $F$.
\item Addition is associative,
$$
x+(y+z)=(x+y)+z
$$
for all $x, y$, and $z$ in $F$.
\item There is a unique element 0 (zero) in $F$ such that $x+0=x$, for every $x$ in $F$.
\item To each $x$ in $F$ there corresponds a unique element $(-x)$ in $F$ such that $x+(-x)=0$.
\item Multiplication is commutative,
$$
x y=y x
$$
for all $x$ and $y$ in $F$.
\item Multiplication is associative,
$$
x(y z)=(x y) z
$$
for all $x, y$, and $z$ in $F$.
\item There is a unique non-zero element 1 (one) in $F$ such that $x 1=x$, for every $x$ in $F$.
\item To each non-zero $x$ in $F$ there corresponds a unique element $x^{-1}$ (or $1 / x)$ in $F$ such that $x x^{-1}=1$.
\item Multiplication distributes over addition; that is, $x(y+z)=$ $x y+x z$, for all $x, y$, and $z$ in $F$.
\end{enumerate}
\end{definition}
\begin{definition}
If $F$ is a field, it may be possible to add the unit 1 to itself a finite number of times and obtain 0:
\begin{equation}
1 + 1 + . . . + 1  = O.
\end{equation}
That does not happen in the complex number field (or in any subfield thereof) . If it does happen in F, then the least $n$ such that the sum of $n$ 1's is 0 is called the \textbf{characteristic} of the field F. If it does not happen
in $\mathbb{F}$, then (for some strange reason) F is called a field of characteristic
zero. Often, when we assume F is a subfield of C, what we want to guarantee
is that F is a field of characteristic zero
\end{definition}
\end{document}