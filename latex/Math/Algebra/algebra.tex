\documentclass[main.tex]{subfiles}
\begin{document}
Algebra is the manipulation of symbols without (necessarily) regard for their meaning, especially in a way that can be formalized in cartesian logic. It is often seen as dual to geometry. While modern algebra has ties and applications nearly everywhere in mathematics, traditionally closest ties are with the number theory and algebraic geometry.
The word 'algebra' is often also used for an algebraic structure:
- often by default an associative unital algebra;
- more generally a monoid object;
- more generally in a different way, a nonassociative algebra;
- an algebra over an operad, of a monad, a PROP, etc;
- an algebra for an endofunctor;
- a model of any algebraic theory. or anything studied in universal algebra;
- higher categorical analogues, many object/family versions of algebras, for example algebroids, and pseudoalgebras (or 2-algebras) over pseudomonads (or 2-monads).

Various fields of mathematics or mathematical concepts can be manipulated in an algebraic or symbolic way, and such approaches or formalized subfields have names like categorical algebra, homological algebra, homotopical algebra and so on. Methods of combinatorics which involve much algebra, and manipulations with formal power series in particular, are called algebraic combinatorics
\end{document}