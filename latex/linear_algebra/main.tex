\documentclass{book}
\usepackage[utf8]{inputenc}
\usepackage{amsmath, amsthm}
\usepackage{physics} %This gives the option to use \norm{}
\usepackage{multirow} %allows merging rows

%This gives and option to combine \textbf{\textsc{...}}
\usepackage[T1]{fontenc}
\usepackage{amsfonts}


% This command get rid of the first figure of the section number coming from the chapter. For example, changes chapter2 section 1 from 2.1 to 1. 
\renewcommand{\thesection}{\arabic{section}}

% \renewcommand{\thesubsection}{\arabic{subsection}}



\newtheorem{thm}{Theorem}[section]
\newtheorem{prop}{Proposition}[section]
\newtheorem{corollary}{Corollary}[theorem]
\newtheorem{lemma}{Lemma}[theorem]



\theoremstyle{remark}
\newtheorem*{note}{Note}
\newtheorem{remark}{Remark}[section]
\newtheorem{example}{Example}[section]


\theoremstyle{definition}
\newtheorem{definition}{Definition}[section]

\newcommand{\sgn}{\text{sgn\,}}
\newcommand{\inv}[1]{{#1}^{-1}}



\title{Set and Quantifiers}
\author{Haoyu}
\date{\today}


\begin{document}

\section{Basic Logical Quantifiers}

\begin{definition}
    A \textbf{math statement} is a statement that is either \textbf{True} or \textbf{False}. For example:
    \begin{itemize}
        \item "An elephant is an animal" is True.
        \item "A dandelion is an animal" is False.
        \item "$1+1=2$" is True.
        \item "$3 \le 2$" is False.
    \end{itemize}
\end{definition}

\begin{definition}
    "$\neg X$" is the negation of statement $X$ (namely "not $X$"). The statement "$\neg X$" is True \textit{iff} (if and only if) statement $X$ is False. For example:
    \begin{itemize}
        \item "An elephant is an animal", which is True.\\
        Negation: "An elephant is not an animal", which is False.
        \item "A dandelion is an animal", which is False.\\
        Negation: "A dandelion is not an animal", which is True.
    \end{itemize}
\end{definition}

\begin{definition} \textbf{Conjunction (and) & Alternative (or)}\\
    The statement "$X \wedge Y$" (read as "$X$ and $Y$") is True \textit{iff} $X$ and $Y$ are both True.\\
    The statement "$X \vee Y$" (read as "$X$ or $Y$") is True \textit{iff} at least one of $X$, $Y$ are True
\end{definition}

\begin{definition}
    The statement "$X \Rightarrow Y$" (read as "statement $X$ \textbf{implies} statement $Y$") means that "if statement $X$ is True, then statement $Y$ is also True".
\end{definition}

\begin{remark} The relationship of $X$, $Y$, and $X \Rightarrow Y$ is shown in the following "\textbf{truth table}":
    \begin{center}
    \begin{tabular}{c|c|c}
         $X$ & $Y$ & $X \Rightarrow Y$  \\
         \hline
         T & T & T \\
         T & F & F \\
         F & T & T \\
         F & F & T
    \end{tabular}
    \end{center}
\end{remark}

\begin{remark}
    Note that if $X$ is False, then $X \Rightarrow Y$ is always True regardless of $Y$. Therefore, when we want to prove $X \Rightarrow Y$, we only need to assume $X$ is True, and then prove that $Y$ is True under such assumption.\\
\end{remark}

\begin{remark}
    Note that $X \Rightarrow Y$ is False only when $X$ is True and $Y$ is False. Therefore, the negation of "$X \Rightarrow Y$" is "$X \wedge (\neg Y)$", namely "$X$ and not $Y$".\\
\end{remark}
    
\begin{remark}
    The statement "$X \Rightarrow Y$" is equivalent to "$\neg Y \Rightarrow \neg X$". The latter is called a \textbf{contra-positive} of "$X \Rightarrow Y$". Therefore, when we want to prove $X \Rightarrow Y$, we can instead prove its contra-positive.
    \begin{center}
    \begin{tabular}{c|c|c|c|c|c}
         $X$ & $Y$ & $X \Rightarrow Y$ & $\neg Y$ & $\neg X$ & $\neg Y \Rightarrow \neg X$ \\
         \hline
         T & T & T & F & F & T\\
         T & F & F & T & F & F\\
         F & T & T & F & T & T\\
         F & F & T & T & T & T
    \end{tabular}
    \end{center}
\end{remark}

\begin{definition}
    The statement "$X \Leftrightarrow Y$" (read as "$X$ \textit{iff} $Y$" or "$X$ is \textbf{equivalent} to $Y$") is True \textit{iff} "$X \Rightarrow Y$ and $Y \Rightarrow X$" is True.
\end{definition}

\begin{remark}
    By definition, when we want to prove $X \Leftrightarrow Y$, we need to prove $X \Rightarrow Y (\Rightarrow$, namely "forward direction"$)$ and $Y \Rightarrow X (\Leftarrow$, namely "backward direction"$)$.
\end{remark}

\begin{definition}
    \textbf{Universal quantifier: $\forall$}\\
    The statement "$\forall$ ($x$ in some range $X$), (statement $Z$)" is True \textit{iff} for all objects $x$ in the specified range, statement $Z$ is True. For example:
    \begin{itemize}
        \item "$\forall $ natural number $x$, $x\ge 0$" is True because all natural numbers are greater or equal to 0.
    \end{itemize}
\end{definition}

\begin{definition}
    \textbf{Existential quantifier: $\exists$}\\
    The statement "$\exists$ ($x$ in some range $X$), such that (statement $Z$)" is True \textit{iff} there exists at least one object $x$ in the specified range such that statement $Z$ is True. For example:
    \begin{itemize}
        \item "$\exists$ natural number $x$, s.t. $x>2$" is True because $x=3$ is a natural number that makes $x>2$.
    \end{itemize}
\end{definition}

\begin{remark} \textbf{Negations of universal and existential statements}\\
    \begin{itemize}
        \item The negation of "$\forall$ ($x$ in range $X$), (statement $Z$)" is "$\exists$ ($x$ in range $X$), s.t. ($\neg Z$)". Therefore, to prove that a universal statement is False, we need to find a counter example. For example:\\
        
        The statement "$\forall$ elephants on Mars, that elephant is purple" is True because we cannot find any counter examples since there are no elephants on Mars (at least for now).
        
        \item The negation of "$\exists$ ($x$ in range $X$), such that (statement $Z$)" is "$\forall$ ($x$ in range $X$), ($\neg Z$)".

        For example, in order to prove that "$\exists$ elephant on Earth, s.t. that elephant is purple" is False, we need to check that all elephants on Earth are not purple, namely to prove "$\forall$ elephant on Earth, that elephant is not purple".
    \end{itemize}
    
\end{remark}



\section{Set}
A set is a list of distinct objects enclosed by curly brackets "\{\}". The set itself is usually denoted with capital letters. For example, $A=\{1, 2, 3, 4\}$ is the set containing 1, 2, 3, and 4. 

\begin{definition}
    If an object is in a set, then it is an \textbf{element} of that set, denoted with the symbol "$\in$". In the previous example, 2 is an element of $A$, thus we say "$2\in A$". The symbol "$\notin$" is used to denote that an object is \textbf{not an element of} a set. In the previous example, 0 is not an element of $A$, thus we say "$0\notin A$".
\end{definition}

\begin{remark}
    By convention, the following set of numbers are denoted with special symbols:
    \begin{itemize}
        \item $\mathbb{N}$ is the set containing all naturally numbers.
        \item $\mathbb{Z}$ is the set containing all integers.
        \item $\mathbb{Q}$ is the set containing all rational numbers.
        \item $\mathbb{R}$ is the set containing all real numbers.
    \end{itemize}
\end{remark}

\begin{definition}
    A set containing no elements is a called a \textbf{trivial set} or an \textbf{empty set}, denoted as $\emptyset = \{\}$.
\end{definition}

\begin{definition}
    If all elements of a set $X$ are also element of another set $Y$, then we say $X$ is a \textbf{subset} of $Y$, denoted as "$X \subseteq Y$". If any element of $X$ is not element of $Y$, then $X$ is not a subset of $Y$, denoted as $X \not\subseteq Y$. Namely:
    \begin{displaymath}
        X \subseteq Y \Leftrightarrow (\forall x \in X, x \in Y)
    \end{displaymath}
\end{definition}

\begin{definition}
    $X$ and $Y$ represent the same set \textit{iff}  $X$ and $Y$ are subset of each other. Namely, $X=Y \Leftrightarrow$  $(X\subseteq Y \hspace{2} and \hspace{3} Y \subseteq X)$.
\end{definition}

\begin{definition}
    set operators\\
    For any sets $X$ and $Y$:
    \begin{itemize}
        \item $X\cup Y$ (read as "X union Y") is the set containing all elements that is element of $X$ \textbf{OR} element of $Y$.
        \item $X\cap Y$ (read as "X intersect Y") is the set containing all elements that is element of $X$ \textbf{AND} element of $Y$.
        \item $X-Y$ (read as "X minus Y") is the set containing all elements that is element of $X$ and \textbf{NOT} element of $Y$.
    \end{itemize}
\end{definition}

\begin{remark}
    \textbf{Set-building notation}\\
    The set $X=\{x$ in some range $|$ statement about $x\}$ ("$:$" can be used to replace "$|$") contains all objects $x$ in the specified range that makes the specified statement True.\\
    For example, the set $X=\{x \in \mathbb{N} : \exists k \in \mathbb{N}, \hspace{2 } s.t. \hspace{3} x=2k\}$ is the set containing all natural numbers that are even.
\end{remark}

\section{Functions}

\begin{definition}
    Intuitively, a function $f$ from a set $A$ to set $B$, written as $f: A \rightarrow B$, maps elements $a \in A$ to a single element $f(a) \in B$. In this case, $A$ is called the \textbf{domain} of $f$, and $B$ is called the \textbf{co-domain} of $f$. \\
    The set $A \times B$ is the set of all ordered pairs $(a, b)$, where $a \in A$ and $b \in B$. Formally speaking, $f$ is a subset of $A \times B$. Thus, $(a, b) \in f$ is another way to represent $f(a) = b$.
\end{definition}

\begin{remark}
    A function $f: A \rightarrow B$ is \textbf{well-defined} \textit{iff} it satisfies both of the following:

    \begin{itemize}
        \item $\forall \hspace{2} a \in A, \exists \hspace{2} b \in B \hspace{3} s.t. \hspace{3} (a, b) \in f$, namely all elements of $A$ are mapped to some elements of $B$ by $f$.
        \item $\forall \hspace{2} b' \in B, b' \ne b \Longrightarrow (a, b') \notin f$, namely each element in $A$ is mapped to \textbf{exactly one} element in $B$ by $f$.
    \end{itemize}

    All functions must be well-defined in their specific domains and co-domains. Therefore, in future discussion we can assume all functions are well-defined, and when we want to defined a new function, we need to make sure the it is well-defined.
\end{remark}

\begin{definition}
    A function $f: A \rightarrow B$ is an \textbf{injection} (injective/one-to-one) \textit{iff} $\forall \hspace{2} a, a' \in A, a \ne a' \Longrightarrow f(a) \ne f(a')$, namely each element in $A$ is mapped to a \textbf{unique} element in $B$ by $f$.
\end{definition}

\begin{definition}
    A function $f: A \rightarrow B$ is a \textbf{surjection} (surjective/onto) \textit{iff} $\forall \hspace{2} b \in B, \exists \hspace{2} a \in A \hspace{3} s.t. \hspace{3} f(a) = b$, namely all elements in $B$ are mapped to by $f$.
\end{definition}

\begin{definition}
    A function $f: A \rightarrow B$ is a \textbf{bijection} \textit{iff} it is both injective and surjective.
\end{definition}

\end{document}