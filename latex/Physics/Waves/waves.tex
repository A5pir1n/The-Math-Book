\documentclass[main.tex]{subfiles}
\begin{document}
\chapter{Harmonic Oscillator}


\chapter{Normal Modes}
\section{Normal Modes}
If there is only one degree of freedom, then both $X$ and $M^{-1}$ are just numbers and the solutions to the equation of motion, (3.62), have the form of a constant amplitude times an exponential factor. In fact, we saw that this form is related to a very general fact about the physics - time translation invariance, (1.33). The arguments of chapter 1 , (1.71)-(1.85), did not depend on the number of degrees of freedom. Thus they show that here again, we can find irreducible solutions, that go into themselves up to an overall constant when the clocks are reset. As in chapter 1, the first step is to allow the solutions to be complex. That is, we replace $(3.62)$ by
$$
\frac{d^2 Z}{d t^2}=-M^{-1} K Z,
$$
where $Z$ is a complex $n$ vector with components, $z_j$. The real parts of the components of $Z$ are the components of a real solution satisfying (3.62),
$$
x_j=\operatorname{Re} z_j .
$$
We will say that the real vector, $X$, is the real part of the complex vector, $Z$,
$$
X=\operatorname{Re} Z,
$$
if (3.64) is satisfied.
Just as in chapter 1, we know that we can find irreducible solutions that have the same form up to an overall constant when the clocks are reset. We know from (1.85) that these have the form
$$
Z(t)=A e^{-i \omega t}
$$
where $A$ is some constant $n$-vector and the angular frequency, $\omega$, is still just a number. Now if $t \rightarrow t+a$,
$$
Z(t) \rightarrow Z(t+a)=e^{-i \omega a} Z(t)
$$
While the irreducible form, (3.66), comes just from time translation invariance, we must still look at the equations of motion to determine the vector, $A$ and the angular frequency, $\omega$. Inserting (3.66) into (3.63), doing the differentiation and canceling the exponential factors from both sides, we find that (3.66) is a solution if
\begin{equation}
    \label{eq:normal_mode_eigen_value_equation}
\omega^2 A=M^{-1} K A .
\end{equation}

This matrix equation is an eigenvalue equation of the form that we discussed in (3.51)-(3.57). $\omega^2$ is the eigenvalue of the matrix $M^{-1} K$ and $A$ is the corresponding eigenvector. Let us see what it means physically.

The real part of the column vector $Z$ specifies the displacement of each of the degrees of freedom of the system. The eigenvalue equation, (3.68), does not involve any complex numbers (because we have not put in any damping). Therefore (as we will see explicitly below), we can choose the solutions so that all the components of $A$ are real. Then the real part of the complex solutions we seek in (3.66) is
$$
X(t)=A \cos \omega t,
$$
or in terms of the components of $A$,
$$
\begin{gathered}
A=\left(\begin{array}{c}
a_1 \\
a_2 \\
\vdots
\end{array}\right) \\
x_1(t)=a_1 \cos \omega t, \quad x_2(t)=a_2 \cos \omega t, \quad \text { etc. }
\end{gathered}
$$
Not only does everything move with the same frequency, but the ratios of displacements of the individual degrees of freedom are fixed. Everything oscillates in phase. The only difference between the motion of the different degrees of freedom is their different amplitudes from the different components of $A$.

The point is worth repeating. Time translation invariance and linearity imply that we can always find irreducible solutions, (3.67), in which all the degrees of freedom oscillate with the same frequency. The extra piece of information that leads to (3.69) is dynamical. If there is no damping, then all the components of $A$ can be chosen to be real, and all the degrees of freedom oscillate not only with the same frequency, but also with the same phase.

If such a solution is to satisfy the equations of motion, then the acceleration must also be proportional to $A$, so that the individual displacements don't get out of synch. But that is what (3.68) is telling us. $-M^{-1} K$ is the matrix that, acting on the displacement, gives the acceleration. The eigenvalue equation (3.68) means that the acceleration is proportional to $A$ again. The constant of proportionality, $\omega^2$, is the return force per unit displacement per unit mass for the particular displacement specified by $A$.

We have already discussed the mathematical structure of the eigenvalue equation in (3.51)-(3.57). We will do it again, for emphasis, in the case of physical interest, (3.68). It should be clear that not every value of $A$ and $\omega^2$ gives a solution of (3.68). We will solve for the allowed values by first finding the possible values of $\omega^2$ and then finding the corresponding values of $A$. To find the eigenvalues, note that (3.68) can be rewritten as
$$
\left[M^{-1} K-\omega^2 I\right] A=0,
$$
where $I$ is the $n \times n$ identity matrix. (3.72) is just a compact way of representing $n$ homogeneous linear equations in the $n$ components of $A$ where the coefficients depend on $\omega^2$. We saw in (3.47) and (3.48) that for systems of $n$ homogeneous linear equations in $n$ unknowns, a nonzero solution exists if and only if the determinant of the coefficient matrix vanishes. The reason is that if the determinant were nonzero, then the matrix, $M^{-1} K-\omega^2 I$, would have an inverse, and we could use (3.31) to conclude that the only solution for the vector, $A$, is $A=0$. Thus to have a nonzero amplitude, $A$, we must have
$$
\operatorname{det}\left[M^{-1} K-\omega^2 I\right]=0 .
$$
(3.73) is a polynomial equation for $\omega^2$. It is an equation of degree $n$ in $\omega^2$, because the term in the determinant from the product of all the diagonal elements of the matrix contains a piece that goes as $\left[\omega^2\right]^n$. All the coefficients in the polynomial are real. Physically, we expect all the solutions for $\omega^2$ to be real and positive whenever the system is in stable equilibrium because we expect such systems to oscillate. Mathematically, we can show that $\omega^2$ is always real, so long as all the masses are positive. We will do this below in (3.127)-(3.130).

Negative $\omega^2$ are associated with unstable equilibrium. For example, consider a mass at the end of a rigid rod, free to swing in the earth's gravitational field in a vertical plane around a frictionless pivot, as shown in figure 3.5. The mass can move along the dotted line. The stable equilibrium position is indicated by the solid line. The unstable equilibrium position is indicated by the dashed line.

When the mass is at the unstable equilibrium point, the smallest disturbance will cause it to fall. Once away from equilibrium, the displacement increases exponentially until the angle from the vertical becomes so large that the nonlinearities in the equation of motion for this system take over. We will discuss this nonlinear oscillator further in appendix B.

Once we have found the possible values of $\omega^2$, we can put each one back into (3.72) to get the corresponding $A$. Because (3.72) is \hyperlink{homogeneous}{homogeneous}, the overall scale of $A$ is not determined, but all the ratios, $a_j / a_k$, are fixed for each $\omega^2$.

\subsection{Normal Modes and Frequencies}
The vector $A$ is called the "normal mode" of the system associated with the frequency $\omega$. Because $A$ is real, in the absence of friction, the complex solutions, (3.66), can be put together into real solutions, like $(3.69)$. The general real solution is of the form
$$
\begin{gathered}
X(t)=\operatorname{Re}[(b+i c) Z(t)]= \\
b A \cos \omega t+c A \sin \omega t=d A \cos (\omega t-\theta)
\end{gathered}
$$
where $b$ and $c$ (or $d$ and $\theta$ ) are real numbers.
We can now construct the complete solution to the equation of motion. Because of linearity, we get it by adding together all the normal mode solutions with arbitrary coefficients that must be set by the initial conditions.

We can now see that the number of different normal modes is always equal to $n$, the number of degrees of freedom. Label the normal modes as $A^\alpha$, where $\alpha$ is a label that (we will argue below) goes from 1 to $n$. Label the corresponding frequencies $\omega_\alpha$. Then the most general possible motion of the system is a sum of all the normal modes,
$$
Z(t)=\sum_{\alpha=1}^n w_\alpha A^\alpha e^{-i \omega_\alpha t}
$$

or in real form (with $w=b+i c$ )
$$
\begin{gathered}
X(t)=\sum_{\alpha=1}^n\left[b_\alpha A^\alpha \cos \left(\omega_\alpha t\right)+c_\alpha A^\alpha \sin \left(\omega_\alpha t\right)\right] \\
=\sum_{\alpha=1}^n d_\alpha A^\alpha \cos \left(\omega_\alpha t-\theta_\alpha\right)
\end{gathered}
$$
where $b_\alpha$ and $c_\alpha$ (or $d_\alpha$ and $\theta_\alpha$ ) are real numbers that must be determined from the initial conditions of the system. Note that the set of all the normal mode vectors must be "complete," in the mathematical sense that any possible configuration of this system can be described as a linear combination of normal modes. Otherwise, we could not satisfy arbitrary initial conditions with the solution, (3.76). This can be proved mathematically (because the matrix, $K$, is symmetric and the masses are positive), but the physical argument will be enough for us here. Likewise no normal mode can possibly be a linear combination of the other normal modes, because each corresponds to an independent possible motion of the physical system with its own frequency. The mathematical way of saying this is that the set of all the normal modes is "linearly independent."
Because the set of normal modes must be both complete and linearly independent, there must be precisely $n$ normal modes, where again, $n$ is the number of degrees of freedom.
If there were fewer than $n$ normal modes, they could not possibly describe all possible configurations of the $n$ degrees of freedom. If there were more than $n$, they could not be linearly independent $n$ dimensional vectors. At least one of them could be written as a linear combination of the others. As we will see later, (3.77) is the physical principle behind Fourier analysis.

If there were fewer than $n$ normal modes, they could not possibly describe all possible configurations of the $n$ degrees of freedom. If there were more than $n$, they could not be linearly independent $n$ dimensional vectors. At least one of them could be written as a linear combination of the others. As we will see later, (3.77) is the physical principle behind Fourier analysis.

It is worth noting that solving the eigenvalue equation, (3.68), gets hard very rapidly as the number of degrees of freedom increases. First you have to compute the determinant of an $n \times n$ matrix. If all the entries are nonzero, this requires adding up $n$ ! terms. Once you have finished that, you still have to solve a polynomial equation of degree $n$. For $n>3$, this cannot be done analytically except in special cases.

On the other hand, it is always straightforward to check whether a given vector is an eigenvector of a given matrix and, if so, to compute the eigenvalue. We will use this fact in the problems at the end of the chapter.

\subsection{The $2 \times 2$ Example}
Let us return to the example from the beginning of this chapter in the special case where the two pendulum blocks have the same mass, $m_1=m_2=m$. Simple as it is, this will be a very important system for our understanding of wave phenomena. 

We should be able to use our usual methods for solving eigenvalue problems to solve for the eigenvalues, which are allowed frequencies in our system, and the corresponding eigenvectors, which are our A vectors, the normal modes. From (3.7) and (3.8), the K matrix has the form: 

$$
K=\left(\begin{array}{cc}
m g / \ell+\kappa & -\kappa \\
-\kappa & m g / \ell+\kappa
\end{array}\right) \text {. }
$$
The $M$ matrix is
$$
M=\left(\begin{array}{cc}
m & 0 \\
0 & m
\end{array}\right) .
$$
Here we solve the eigenvalue problem of equation \ref{eq:normal_mode_eigen_value_equation} 
\begin{equation*}
    \label{eq:normal_mode_eigen_value_equation}
\omega^2 A=M^{-1} K A .
\end{equation*}
and get that the angular frequencies of the normal modes are
$$
\omega_1^2=g / \ell, \quad \omega_2^2=g / \ell+2 \kappa / m.
$$




\end{document}