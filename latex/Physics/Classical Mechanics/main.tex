\documentclass{article}
\usepackage[utf8]{inputenc}
\usepackage{amsmath,amssymb,amsthm}

\title{Theoretical Mechanics Notes}
\author{Hongyu}
\date{December 2020}

\begin{document}

\maketitle

\section{The equations of motion}
\subsection{The principle of Least action}
The \textbf{principle of least action} or \textbf{Hamilton's principle} states that: if the system occupy, at instants $t_1$ and $t_2$, positions defined by two sets of values of the coordinates, $q^(1)$ and $q^(2)$. Then the systems moves in the way such that the integral \begin{equation}
    S = \int_{t_1}^{t_2}L(q,\dot q,t)dt
\end{equation} takes the lest possible value. And the function $L$ is called the Lagrangian of the system, the integral is called the \textit{action}.
\end{document}
