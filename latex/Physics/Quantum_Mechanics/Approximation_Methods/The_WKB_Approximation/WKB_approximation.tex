\documentclass[main.tex]{subfiles}
\begin{document}
\chapter{THE WKB APPROXIMATION}
\section{Introduction}
The WKB (Wentzel, Kramers, Brillouin) \footnote{In Holland it's KWB. in France it's BWK, and in England it's JWKB (for Jeffreys).} method is a technique for obtaining approximate solutions to the time-independent Schrödinger equation in one dimension (the same basic idea can be applied to many other differential equations, and to the radial part of the Schrödinger equation in three dimensions). It is particularly useful in calculating bound state energies and tunneling rates through potential barriers.

The essential idea is as follows: Imagine a particle of energy $E$ moving through a region where the potential $V(x)$ is constant. If $E>V$, the wave function is of the form
$$
\psi(x)=A e^{\pm i k x}, \quad \text { with } k \equiv \sqrt{2 m(E-V)} / \hbar .
$$
The plus sign indicates that the particle is traveling to the right, and the minus sign means it is going to the left (the general solution, of course, is a linear combination of the two). The wave function is oscillatory, with fixed wavelength $(\lambda=2 \pi / k)$ and unchanging amplitude $(A)$. Now suppose that $V(x)$ is not constant, but varies rather slowly in comparison to $\lambda$, so that over a region containing many full wavelengths the potential is essentially constant. Then it is reasonable to suppose that $\psi$ remains practically sinusoidal, except that the wavelength and the amplitude change slowly with $x$. This is the inspiration behind the WKB approximation. In effect, it identifies two different levels of $x$-dependence: rapid oscillations, modulated by gradual variation in amplitude and wavelength.

By the same token, if $E<V$ (and $V$ is constant), then $\psi$ is exponential:
$$
\psi(x)=A e^{\pm \kappa x} . \quad \text { with } \kappa \equiv \sqrt{2 m(V-E)} / \hbar .
$$
And if $V(x)$ is not constant, but varies slowly in comparison with $1 / \kappa$, the solution remains practically exponential, except that $A$ and $\kappa$ are now slowly-varying functions of $x$.

Now, there is one place where this whole program is bound to fail, and that is in the immediate vicinity of a classical turning point, where $E \approx V$. For here $\lambda$ (or $1 / \kappa$ ) goes to infinity, and $V(x)$ can hardly be said to vary "slowly" in comparison. As we shall see, a proper handling of the turning points is the most difficult aspect of the WKB approximation, though the final results are simple to state and easy to implement.

\section{THE "CLASSICAL" REGION}
The Schrödinger equation,
$$
-\frac{\hbar^2}{2 m} \frac{d^2 \psi}{d x^2}+V(x) \psi=E \psi,
$$
can be rewritten in the following way:
$$
\frac{d^2 \psi}{d x^2}=-\frac{p^2}{\hbar^2} \psi,
$$
where
$$
p(x) \equiv \sqrt{2 m[E-V(x)]}
$$
is the classical formula for the momentum of a particle with total energy $E$ and potential energy $V(x)$. For the moment, I'll assume that $E>V(x)$, so that $p(x)$ is real; we call this the "classical" region, for obvious reasons-classically the particle is confined to this range of $x$ (see Figure 8.1). In general, $\psi$ is some complex function; we can express it in terms of its amplitude, $A(x)$, and its phase, $\phi(x)$-both of which are real:
$$
\psi(x)=A(x) e^{i \phi(x)} .
$$
Using a prime to denote the derivative with respect to $x$, we find:
$$
\frac{d \psi}{d x}=\left(A^{\prime}+i A \phi^{\prime}\right) e^{i \phi} .
$$
and
$$
\frac{d^2 \psi}{d x^2}=\left[A^{\prime \prime}+2 i A^{\prime} \phi^{\prime}+i A \phi^{\prime \prime}-A\left(\phi^{\prime}\right)^2\right] e^{i \phi}
$$
Putting this into Equation 8.1:
$$
A^{\prime \prime}+2 i A^{\prime} \phi^{\prime}+i A \phi^{\prime \prime}-A\left(\phi^{\prime}\right)^2=-\frac{p^2}{\hbar^2} A .
$$
This is equivalent to two real equations, one for the real part and one for the imaginary part:
$$
A^{\prime \prime}-A\left(\phi^{\prime}\right)^2=-\frac{p^2}{\hbar^2} A, \quad \text { or } \quad A^{\prime \prime}=A\left[\left(\phi^{\prime}\right)^2-\frac{p^2}{\hbar^2}\right],
$$
and
$$
2 A^{\prime} \phi^{\prime}+A \phi^{\prime \prime}=0 . \quad \text { or } \quad\left(A^2 \phi^{\prime}\right)^{\prime}=0 .
$$
Equations $8.6$ and $8.7$ are entirely equivalent to the original Schrödinger equation. The second one is easily solved:
$$
A^2 \phi^{\prime}=C^2 . \quad \text { or } \quad A=\frac{C}{\sqrt{\phi^{\prime}}},
$$
where $C$ is a (real) constant. The first one (Equation 8.6) cannot be solved in general-so here comes the approximation: We assume that the amplitude A varies slowly, so that the $A^{\prime \prime}$ term is negligible. (More precisely, we assume that $A^{\prime \prime} / A$ is much less than both $\left(\phi^{\prime}\right)^2$ and $p^2 / \hbar^2$.) In that case we can drop the left side of Equation 8.6, and we are left with
$$
\left(\phi^{\prime}\right)^2=\frac{p^2}{\hbar^2}, \quad \text { or } \quad \frac{d \phi}{d x}=\pm \frac{p}{\hbar} .
$$

and therefore
$$
\phi(x)=\pm \frac{1}{\hbar} \int p(x) d x
$$
(I'll write this as an indefinite integral, for now-any constants can be absorbed into $C$, which may thereby become complex.) It follows that
$$
\psi(x) \cong \frac{C}{\sqrt{p(x)}} e^{\pm \frac{i}{\hbar} \int p(x) d x}
$$
and the general (approximate) solution will be a linear combination of two such terms, one with each sign.
Notice that
$$
|\psi(x)|^2 \cong \frac{|C|^2}{p(x)}
$$
which says that the probability of finding the particle at point $x$ is inversely proportional to its (classical) momentum (and hence its velocity) at that point. This is exactly what you would expect - the particle doesn't spend long in the places where it is moving rapidly, so the probability of getting caught there is small. In fact, the WKB approximation is sometimes derived by starting with this "semi-classical" observation, instead of by dropping the $A^{\prime \prime}$ term in the differential equation. The latter approach is cleaner mathematically, but the former offers a more plausible physical rationale.

\begin{example}

\end{example}
\end{document}